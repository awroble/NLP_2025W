\documentclass[11pt]{article}

% Change "review" to "final" when you want the final version.
\usepackage[final]{acl}
\usepackage{placeins}
\usepackage{times}
\usepackage{latexsym}
\usepackage{float}
\usepackage{rotating}
\usepackage{tabularx}

\usepackage{caption}


% Encoding (UTF-8)
\usepackage[T1]{fontenc}
\usepackage[utf8]{inputenc}

% Recommended extras (safe in ACL template)
\usepackage{microtype}
\usepackage{url}
\usepackage{booktabs}
\usepackage{tabularx}
\usepackage{xurl}

\title{International Agreements Data Base Mining}

\author{
  Paulina Kulczyk \\
  Warsaw University of\\
  Technology \\
  \texttt{01151498@domain} 
  \And
  Mateusz Wiktorzak \\
  Warsaw University of \\
  Technology \\
  \texttt{01171275@pw.edu.pl}
  \And
  Muhamed Fahim Asim \\
  Warsaw University of \\
  Technology \\
  \texttt{01205609@pw.edu.pl}
  \AND
  Mateusz Zagórski \\
  Warsaw University of Technology \\
  \texttt{01161597@pw.edu.pl}
  \And
  Supervisor: Anna Wróblewska \\
  Warsaw University of Technology \\
  \texttt{anna.wroblewska1@pw.edu.pl}
}
\setlength\titlebox{8cm}  % try 8cm, 9cm, 10cm until there is no overlap

\begin{document}
\maketitle


\begin{abstract}

Research on subnational diplomacy is often hindered by the inaccessibility of international legal agreements signed by U.S. states and cities, which predominantly exist as low-quality, unstructured PDF scans. This project addresses the challenge of converting these degraded documents into a reliable, structured dataset through a comprehensive information extraction pipeline. We target thirteen specific data points, ranging from the identification of parties and areas of cooperation to the analysis of recurring clauses and validity terms. The methodology integrates a review of state-of-the-art techniques—including advanced Optical Character Recognition (OCR), legal-domain Named Entity Recognition (NER), and semantic deduplication using Legal-SBERT—with a rigorous dataset analysis to address noise and linguistic heterogeneity. Following a proof-of-concept evaluation of these methods on a corpus subset, we present a modular, scalable pipeline designed to handle OCR noise, classify agreement types, and extract metadata with high accuracy. The resulting system enables reproducible large-scale processing, providing a foundational tool for quantitative research into the legal frameworks of subnational international relations.

\end{abstract}

\section{Introduction}

This project addresses the challenge of extracting structured and reliable information from a large collection of international legal agreements signed by U.S. states and cities. The source documents primarily exist as low-quality, inconsistently formatted PDF scans, making them difficult to access and analyze using standard computational methods. These agreements represent a valuable resource for research on subnational diplomacy, yet their usability is limited by OCR noise, metadata inconsistencies, duplicate records, and unreliable entity extraction.

The scope of tasks, which were predefined in the beginning of the work and will be referred to later, are:
\begin{enumerate}
    \item Identification of areas of cooperation mentioned in the agreements.
    \item Identification of the parties involved.
    \item Identification of the types of agreements.
    \item Determination of the percentage of agreements under the patronage of Sister Cities International.
    \item Identification of international organizations mentioned in the agreements.
    \item Determination of the terms of validity for each agreement.
    \item Identification of the length of each agreement.
    \item Determination of the conditions for extending each agreement.
    \item Analysis of the frequency of recurring clauses in the agreements.
    \item Identification of the partners with whom the agreements tend to be more detailed.
    \item Indication of whether the agreement includes an evaluation of its implementation.
    \item Identification of whether the agreement mentions any coordination of activities with other entities.
    \item Identification of whether the agreement refers to other legal documents.
\end{enumerate}


The objective of this work is to transform these documents into a clean, analyzable dataset through a comprehensive information extraction pipeline. We first review state-of-the-art methods relevant to this task, including advanced OCR models for degraded scans, legal-domain named entity recognition, agreement-type and clause classification, text segmentation, zero-shot contract analysis, and semantic deduplication using Legal-SBERT and FAISS. This review identifies the most suitable techniques for handling the complexity and noise characteristic of subnational legal agreements.

We then analyze the dataset itself, examining document quality, variability across agreements, metadata completeness, and linguistic heterogeneity. Basic statistical analysis and qualitative observations highlight challenges related to scan degradation, inconsistent formatting, and multilingual content, motivating informed methodological choices.

In the proof-of-concept stage, the most promising techniques are evaluated on a subset of the corpus to assess practical performance under realistic computational constraints. OCR models, NER systems, clause extractors, and similarity-based deduplication methods are tested to determine which approaches deliver acceptable accuracy.

Finally, the implemented methods were ran on a whole dataset and the resulting outputs were analyzed to verify that each task is solved to an acceptable standard.

\section{SOTA Analysis}

\subsection{Optical Character Recognition (OCR)}

Recent work in document understanding shows that OCR alone is insufficient for reliably processing formal government PDFs that mix multi-page scans, complex layouts, stamps, signatures, and legal boilerplate \cite{tesseract2007_smith,documentai2021_survey}. This project targets converting such heterogeneous, often low-quality inputs into structured representations with correct page, paragraph, and reading-order reconstruction so that downstream systems can search, validate, and analyze official records at scale, including measuring agreement length in pages or words. The task lies at the intersection of OCR, layout-aware information extraction, and document-structure recovery, building on multimodal document AI, form understanding, and invoice-style IE for business and government workflows \cite{documentai2021_survey}.

Modern OCR backbones are largely open-source and deep-learning based, with engines such as Tesseract, PaddleOCR, TrOCR, and docTR forming standard building blocks \cite{tesseract2007_smith}. Tesseract remains a widely adopted baseline for printed and multilingual text in large digitization efforts, including legal and governmental archives \cite{tesseract2007_smith}. Newer engines extend these capabilities with convolutional and transformer architectures that jointly model visual and textual context, improving robustness to noise, skew, and varying fonts common in legacy scans and faxed documents and enabling reliable word-level statistics for agreement-length estimation \cite{doctr2021}. Work on degraded and historical documents shows that binarization, denoising, background removal, and super-resolution further boost OCR quality in the presence of stamps, seals, marginal notes, and low-resolution scans \cite{gatos2019_degraded_binarization}.

Beyond OCR, state-of-the-art layout-aware IE models treat documents as multimodal objects where text, layout, and visual signals are processed jointly \cite{documentai2021_survey}. LayoutLM-style architectures pretrain on large corpora of business and administrative documents with 2D positional embeddings and visual features, then fine-tune for key-value extraction, form understanding, and document QA \cite{documentai2021_survey}. Benchmarks such as FUNSD, CORD, SROIE, and LIE show that layout-aware models consistently outperform text-only baselines on entity extraction and relation prediction, which directly supports extracting areas of cooperation from cleaned agreement text as basic semantic fields \cite{documentai2021_survey}. LIE is particularly relevant because it includes product and official documents from more than 150 organizations, with diverse templates and page lengths resembling real-world government PDFs.  At the structural level, work on PDF and document structure extraction focuses on recovering logical hierarchy, section boundaries, and reading order across multi-column and multi-page documents \cite{luz2011_pdf_structure_extraction}. Classical methods such as XY-cut and its variants are strong baselines for segmenting text blocks and columns but are sensitive to noise and layout variation and often require heavy heuristic tuning. Newer graph-based and neural approaches operate over layout graphs to recover headings, and tables of contents, which in this context support reliable paragraph reconstruction and the identification of recurring clause segments that can be standardized and counted across agreements \cite{luz2011_pdf_structure_extraction,documentai2021_survey}. For this project, these techniques motivate a hybrid design that combines robust block segmentation with learning-based ordering and section classification tuned to government document genres.

Commercial systems such as Google Document AI, ABBYY, and Azure Document Intelligence achieve strong extraction accuracy on forms, invoices, and contracts. However, their closed-source nature, dependence on proprietary clouds, and limited transparency about training data and behavior complicate tuning them \cite{documentai2021_survey,msdocai2021}. Open benchmarks such as CC-OCR and evaluations of layout-aware and multimodal models indicate that open-source and academic approaches can match or surpass these systems in adaptability and extensibility, especially with domain-specific fine-tuning \cite{documentai2021_survey}. Collectively, this literature defines the current SOTA and supports a domain-specialized, open pipeline for formal government PDFs that combines proven components while addressing gaps in domain adaptation and full-document reconstruction, including agreement-length measurement, basic extraction of cooperation areas, and clause-frequency analysis over standardized segments.

\subsection{Agreement type, sides, organizations extraction}

There are three interconnected subtasks that can be grouped together based on the similarity of their
outputs:

2. Identify the parties involved,

3. Identify the types of agreements,

5. Identify international organizations mentioned.

These tasks can be addressed using two primary NLP techniques: Named Entity Recognition (NER)
for extracting entities such as parties and organizations (tasks 2. and 5.), and Document Classification
for categorizing the types of agreements (task 3.). This grouping reflects the underlying workflows
needed to transform unstructured legal texts into structured data, performing extraction and finding
the key information.

\subsubsection{Legal-Domain Named Entity Recognition (NER)}

The most advanced model in legal NER is LegNER \cite{karamitsos2025legner}, a domain-adapted transformer model pretrained on extensive legal corpora with span-level supervision. LegNER achieves F1 scores over 99\%, surpassing other models like Legal-BERT by several percentage points. Its architecture, based on BERT-base but fine-tuned on legal-specific language, enables precise recognition of key legal entities such as parties, laws, and organizations crucial for international agreements. The model maintains high precision and recall, making it reliable for clean entity extraction even on complex, noisy legal
data.

Other competitive models include SpanBERT-Legal, which captures entity spans well, and ContractNLI \cite{henderson2021contractnli}, which extends entity recognition to contract clause classification but typically scores
lower than LegNER in benchmarks.

\subsubsection{Legal agreement type classification}

Transformers like RoBERTa, LegalBERT, and T5 are the leading models for legal document classification tasks, including categorizing types of agreements. They are fine-tuned on domain-specific datasets to identify particular categories of contracts. These models leverage multi-label classification and contextual understanding, achieving near-human accuracy in distinguishing diverse legal document types.

\subsubsection{Essential datasets}

1. LEDGAR \cite{tuggener2020ledgar}: A large-scale, multi-label corpus for contract provision classification created from SEC filings (EDGAR). It contains nearly 100,000 provisions across more than 12,000 labels from over 60,000 contracts. LEDGAR is widely used for fine-tuning models like LegalBERT for clause and agreement type classification, supporting scalable legal NLP research,  

2. CUAD (Contract Understanding Atticus Dataset): An expert-annotated dataset with over 2,000 clauses across 41 contract categories from 510 commercial contracts. CUAD enhances classification at the clause-level, useful for detailed contract analysis and extraction,

3. Contract NLI Dataset: A resource for document-level contract classification and clause identification.

These datasets provide rich, annotated data essential for training and benchmarking the state-of-
the-art models in legal NER and contract classification.

\subsection{Deduplication (Legal-SBERT + FAISS)}

Large collections of international agreements often contain duplicate or near-duplicate documents originating from multiple sources, OCR rescans, multilingual versions, or slightly edited reissues. Removing such duplicates is essential for producing reliable statistics, including the percentage of agreements signed under the patronage of Sister Cities International and the identification of partners with whom agreements tend to be more detailed. To address this, the pipeline applies state-of-the-art sentence embedding–based deduplication, combining Legal-SBERT for semantic encoding and FAISS for fast large-scale similarity search.

\subsubsection{Legal-SBERT for semantic encoding}

Legal-SBERT \cite{askari2024retrieval} is a domain-adapted variant of Sentence-BERT fine-tuned on large legal corpora, achieving significant improvements over standard BERT and SBERT in semantic similarity tasks involving contracts, statutes, and judicial documents. Reimers \& Gurevych’s SBERT architecture enables fixed-size embeddings optimized for cosine-similarity retrieval, while legal-domain adaptations such as Legal-SBERT demonstrate state-of-the-art performance on legal sentence similarity benchmarks, with improvements of up to +7–10 points in Spearman correlation. Using Legal-SBERT ensures that semantically identical agreements—despite OCR noise, reformatting, or paraphrasing—receive
nearly identical embeddings, enabling accurate duplicate detection.

\subsubsection{FAISS for scalable similarity indexing}

FAISS (Facebook AI Similarity Search) is the leading library for high-dimensional vector indexing and approximate nearest-neighbor search. It supports millions of embeddings with sub-second lookup times using optimized GPU/CPU indices such as HNSW and IVF-PQ. FAISS is widely used in legal NLP for clustering and deduplication in large corpora (e.g., EDGAR, global treaty databases). In this project, FAISS enables:

\begin{itemize}
\item efficient retrieval of nearest neighbors for each agreement
\item identification of duplicate or near-duplicate records (e.g., cosine similarity > 0.92)
\item removal of redundant documents prior to downstream analysis
\end{itemize}

This step is crucial for ensuring that downstream statistics—such as identifying detailed partner-
ships or calculating Sister Cities International patronage—are not distorted by duplicated agreements
or OCR-generated copies.

\subsection{Semantic Similarity Clustering (UMAP + HDBSCAN)}

Beyond deduplication, semantic similarity clustering enables the discovery of latent patterns in the agreement corpus. This step directly supports tasks such as identifying partners associated with more detailed agreements, detecting coordination with other entities, and clustering agreements that reference other legal documents. These phenomena correspond to distinct semantic regions of the embedding space.

\subsubsection{Dimensionality reduction with UMAP}

Uniform Manifold Approximation and Projection (UMAP) is one of the most advanced nonlinear dimensionality-reduction algorithms for high-dimensional embeddings. Unlike PCA or t-SNE, UMAP preserves both local neighborhoods and global manifold structure, producing representations well suited for density-based clustering. Applied to Legal-SBERT embeddings, UMAP:

\begin{itemize}
\item reduces 768-dimensional vectors to 10–50 dimensions
\item preserves semantic topology of legal texts
\item enhances density separation between agreement types, detailed vs. minimal agreements, refer-
ences to external legal documents, and coordination clauses
\end{itemize}

UMAP has become a standard preprocessing step in legal and scientific document clustering because it maintains structural coherence while enabling more robust cluster detection.

\subsection{Density-based clustering with HDBSCAN}

HDBSCAN \cite{campello2013density} is the state-of-the-art clustering algorithm for noisy text datasets, outperforming k-means and DBSCAN in legal document grouping. It automatically identifies dense regions in the
reduced embedding space and labels sparse regions as “noise,” which is ideal for heterogeneous legal corpora.

For this project, HDBSCAN enables:

\begin{itemize}
\item grouping agreements with similar structural complexity (to identify highly detailed partners)
\item detecting clusters of agreements that reference other legal documents
\item isolating agreements that mention coordination with external entities (e.g., federal agencies,
NGOs, international organizations)
\item discovering thematic clusters without requiring predefined labels
\end{itemize}

In combination, UMAP + HDBSCAN form a robust state-of-the-art pipeline for unsupervised legal document analysis, enabling empirical insight into agreement structure and recurring patterns relevant for social science research.

\subsection{Determine Terms of Validity}

Extracting and normalizing temporal expressions in contracts is essential to identify start and end dates, durations, and validity periods. Temporal information may be explicit or implicit, expressed via phrases such as ”for the term of the project” or ”until December 31, 2026”. Accurate extraction
supports contract interpretation, compliance, and automated reasoning.

Rule-based systems such as HeidelTime remain a strong baseline, achieving high precision on legal corpora such as CUAD and LEDGAR by encoding patterns for dates, durations, and relative expressions \cite{strotgen2010heideltime}.

Neural approaches, particularly LegalBERT fine-tuned for token classification, capture implicit temporal phrasing and domain-specific variations. ContractNLI verification further refines extraction, utilizing hybrid pipelines that combine rule-based normalization, neural token classification, and NLI verification, with F1 scores consistently above 0.85.

\subsection{Determine Conditions for Extending the Agreement}

Renewal and extension clauses define when a contract may be automatically renewed, mutually extended, or optionally extended. They often include nested conditions, notice periods, or triggers, making detection challenging.

Transformer-based models such as LegalBERT and Longformer, fine-tuned on CUAD and LEDGAR, detect renewal clauses and distinguish automatic from optional extensions.

Hybrid pipelines combining neural span extraction, rule-based temporal/condition normalization, and NLI or QA verification improve precision and resolve cross-references, with F1 scores typically
between 0.80 and 0.85. Simple rule-based or keyword approaches can supplement neural methods in less complex contracts.

\subsection{Indicate Whether the Agreement Includes an Evaluation of Its Implementation}

Evaluation clauses monitor compliance or performance, specifying assessment, reporting, or auditing duties. They are often sparse or implicit, with verbs like ”review”, ”assess”, or ”audit”, sometimes
spanning multiple sections. 

Fine-tuned LegalBERT or LegalPro-BERT models detect evaluation clauses in CUAD and LEDGAR, while token-classification extracts roles and reporting frequency. NLI or QA verification captures implicit evaluation obligations, improving precision.

Hybrid pipelines, combining token-classification, rule-based heuristics, and NLI/QA verification, achieved F1 scores around 0.80–0.83, balancing recall and precision.

\subsection{Metadata Reliability}

Metadata for international agreements—such as dates, partner names, issuing authorities, agreement types, and references to Sister Cities International—are often inconsistent or incomplete due to variable formatting, or manual transcription. Ensuring metadata reliability is fundamental for tasks such as computing the percentage of agreements under Sister Cities International and systematically identifying coordination clauses, external references, and detailed agreements with specific partners.

\subsubsection{Metadata normalization}

Normalization consolidates metadata fields into standardized canonical forms. This includes:

\begin{itemize}
\item organization name normalization (e.g., “Sister Cities International,” “Sister City Int’l,” “SCI”) using fuzzy matching and embedding-based similarity partner
\item entity normalization using Wikidata cross-referencing and legal-NER outputs
\item date normalization using ISO-8601 standards
\item agreement type harmonization based on controlled vocabularies from international treaty databases
\end{itemize}

State-of-the-art metadata normalization relies on neural fuzzy-matching models such as Ditto and embedding-based entity alignment, which significantly outperform rule-based approaches by capturing semantic-level similarity across heterogeneous entity representations. However, transformer-based models such as Ditto require substantial computational resources—particularly GPU acceleration and fine-tuning on domain-specific datasets—making them less practical for smaller-scale projects or environments with limited hardware. Consequently, for this study, we adopt a fuzzy string matching approach using RapidFuzz \cite{ye2021rapidfuzz}. This method provides sufficient accuracy for canonicalizing entities
such as variations of ‘Sister Cities International’ while remaining computationally lightweight and feasible within the available CPU resources and project timeline.

\subsubsection{Consistency checks}

After normalization, consistency checks detect contradictions or missing information. Typical checks
include:

\begin{itemize}
\item verifying that agreement dates are chronological
\item confirming that referenced legal documents exist in the database
\item ensuring that coordination entities (ministries, city councils, NGOs) match recognized NER categories
\item checking cross-document consistency for Sister Cities International attribution
\item validating that long agreements (as identified via clustering or word counts) correspond to ex-
pected partner categories
\end{itemize}

Such checks rely on techniques from information quality management \cite{batini2016data} and cross-document validation frequently applied in treaty and contract corpora. In addition to these traditional approaches,
modern LLMs can assist in consistency verification by detecting anomalous or contradictory metadata patterns—such as misaligned partner names, unexpected absence of coordination clauses, or references that do not match the text—providing a hybrid methodology that ensures metadata is reliable, coherent, and robust across the entire corpus.

Together, metadata normalization and consistency checks ensure that extracted insights—such as detailed agreement patterns or coordination with other entities—are based on validated and coherent metadata, increasing reliability for social science research.

\section{Dataset}

\subsection{Data collection and preprocessing}

The raw data were initially acquired in two formats: PDF and HTML, with a distribution ratio of approximately 2:1. During the inspection phase, several critical preprocessing steps were taken:

\begin{itemize}
\item Format Filtering: All HTML files were removed from the dataset. Inspections revealed that these files were low-quality extractions from PDFs, containing fragmented sentences and inconsistent formatting that would impede reliable mining.
\item OCR Processing: Optical Character Recognition (OCR) was applied to the remaining docu-
ments to generate two output types:
\begin{enumerate}
\item .txt files containing the raw agreement text.
\item .json files containing structured statistical summaries of the text.
\end{enumerate}
\item Storage Structure: The processed files are organized into 50 distinct directories, each named
after a US state, stored within a central OCR-output directory.
\end{itemize}

The original data shared with us by the Uniwersytet Łódzki team is stored at this address: \url{https://uniwersytetlodzki-my.sharepoint.com/personal/marcin_frenkel_wsmip_uni_lodz_pl/_layouts/15/onedrive.aspx?id=%2Fpersonal%2Fmarcin%5Ffrenkel%5Fwsmip%5Funi%5Flodz%5Fpl%2FDocuments%2FU%C5%81%2FProjekt%20z%20Politechnik%C4%85%20Warszawsk%C4%85%20%2D%20baza%20um%C3%B3w%2FBaza%20um%C3%B3w&ga=1}.

All preprocessed data can be found at the following address: \url{https://drive.google.com/drive/folders/12lgLmvTZY6Q3Rz2L9JxGmPggIAXDLJ7M?fbclid=IwY2xjawPIwK5leHRuA2FlbQIxMABicmlkETE4WTE3dzRGVFlRUmY1WDVPc3J0YwZhcHBfaWQQMjIyMDM5MTc4ODIwMDg5MgABHikF555zPJ0CW8CP25tOVf89vdwAn_tyd7s-nsh-fzPK306j5YHTIKBhPaGO_aem_REAubR4osJ8cwvnihpEqpA}.

\subsection{Dataset composition}

The current dataset comprises a total of 298 agreements. While the structure accounts for all 50 US states, the actual distribution of documents is as follows:

\begin{itemize}
\item States with Agreements: 29 (58\% coverage).
\item States without Agreements: 21 (42\% coverage).
\item Total Sample Size: 298 documents.
\end{itemize}

\subsection{Statistical Analysis}

The distribution of agreements per state is highly skewed, characterized by a few significant outliers and many states with minimal data.

\begin{table}[t]
\centering
\begin{tabular}{@{}ll@{}}
\toprule
Metric & Value \\
\midrule
Minimum agreements per state & 0 \\
Maximum agreements per state & 116 \\
Average agreements per state & 5.96 \\
Median agreements per state & 1.0 \\
Standard Deviation ($\sigma$) & 18.05 \\
\bottomrule
\end{tabular}
\caption{Dataset Summary Statistics}
\end{table}

\subsection{Geographical outliers}

A small number of states contribute the vast majority of the dataset. The top 5 states by agreement count are shown in Table \ref{tab:top_5_states}, and top 10 in the Figure \ref{fig:barplot_top_10_states}. Notably, California alone accounts for nearly 39\% of the total dataset. Conversely, many states, such as Missouri, Florida, Georgia, and New Mexico, currently contain zero recorded agreements in this specific project iteration.

\begin{figure}[H]
    \centering
    \includegraphics[width=\linewidth]{images/states_doc_count_barplot.png}
    \caption{Barplot with count of agreements for top 10 states}
    \label{fig:barplot_top_10_states}
\end{figure}

\begin{table}[t]
\centering
\begin{tabular}{@{}llll@{}}
\toprule
Rank & State & Count \\
\midrule
1 & California & 116 \\
2 & Texas & 54 \\
3 & Utah & 22 \\
4 & Arizona & 17 \\
5 & Alaska & 12 \\
\bottomrule
\end{tabular}
\caption{Top 5 States by Document Count}
\label{tab:top_5_states}
\end{table}

\subsection{Conclusion}

The dataset is robust in its total document count (298) but exhibits significant geographical imbalance. Preprocessing has successfully filtered out noisy HTML data, ensuring that the remaining OCR-processed text is of high quality for subsequent legal mining and metadata normalization tasks.

\section{Exploratory data analysis}

This section presents the key findings of EDA conducted on the International Agreements Database. The dataset consists of OCR-processed agreement documents organized by U.S. state. The analysis focuses on the spatial distribution, document characteristics, and thematic content of these agreements.

\subsection{Dataset overview and spatial distribution}

The dataset comprises 298 agreements distributed across 50 directories, each corresponding to a U.S. state. The distribution is highly skewed, with only 29 states containing any agreement documents (Figure \ref{fig:barplot_states}).

\begin{itemize}
\item \textbf{Total Volume}: There are 298 agreements in total.
\item \textbf{Geographic Concentration}: The vast majority of agreements are concentrated in California (116 agreements) and Texas (54 agreements). Utah also holds a significant number (22), while many other states (e.g., Virginia, Illinois, New Hampshire) contain only a single document (Figure \ref{fig:barplot_states}).
\begin{figure}[H]
    \centering
    \includegraphics[width=\linewidth]{images/newplot.png}
    \caption{Barplot with count of agreements across all states}
    \label{fig:barplot_states}
\end{figure}
\item \textbf{Coverage}: 21 out of 50 states have no representation in the dataset (Figure \ref{fig:doc_count_map}).
\item \textbf{Outliers}: While California and Texas lead in absolute numbers, it is worth noting that Alaska and Vermont—which are among the least populous states—have a relatively large number of contracts (12 and 11, respectively). We can therefore classify them as outliers in terms of population versus number of contracts. These states appear as notable hubs for international activity relative to their population size, likely due to their geographic proximity to Canada.
\end{itemize}

\begin{figure}[H]
    \centering
    \includegraphics[width=\linewidth]{images/doc_counts_map_dense_tight.png}
    \caption{Map of distribution of documents through states}
    \label{fig:doc_count_map}
\end{figure}

\subsection{Document characteristics}

An analysis of document length (page count and word count) reveals that most agreements are relatively concise, though significant complexity exists in specific regions (Figure \ref{fig:new2}).
\begin{figure}[H]
    \centering
    \includegraphics[width=\linewidth]{images/newplot2.png}
    \caption{Distribution of document length (page count) by state}
    \label{fig:new2}
\end{figure}
\begin{itemize}
\item \textbf{Page Count}: The agreements range from 1 to 21 pages. The distribution is heavily weighted towards shorter documents, with the majority having 5 pages or fewer (typically standard MOUs).
\item Word Count: The total word count per document ranges from 94 to 7,368 words.
\item Variation: While California and Texas have the highest volume, they also show the widest variance in document length. Interestingly, Michigan and Pennsylvania exhibit higher median page counts, suggesting their agreements may involve more complex regulatory or infrastructure frameworks compared to the single-page declarations found in other states. Maximum length documents (21 pages) were identified in California, Vermont, and Alaska. \end{itemize}


\subsection{Thematic analysis and N-gram extraction}

To identify key themes, frequentist analysis was performed on word tokens and bigrams after removing standard English stopwords and state names.

\begin{figure}[H]
    \centering
    \includegraphics[width=\linewidth]{images/newplot4.png}
    \caption{Key thematic bigrams for selected states.}
    \label{fig:new4}
\end{figure}

\subsubsection{Top Words}

The most frequent terms across the corpus highlight the administrative and cooperative nature of the texts. The top words include ”state” (921 occurrences), ”parties” (423), ”information” (422), and ”energy” (380). Terms like ”participants”, ”supervisors”, and ”understanding” are also prominent.

\subsubsection{Bigram analysis}

Bigram analysis reveals specific areas of international cooperation (Figure \ref{fig:new4}). After filtering out generic phrases (e.g., ”United States”), the following key themes emerged:

\begin{itemize}
\item \textbf{Legal Frameworks}: The most common bigram is ”memorandum understanding” (904 occurrences), confirming that MOUs are the primary legal instrument used.
\item \textbf{Environmental Issues}: A significant portion of the corpus focuses on climate and ecology,
evidenced by bigrams such as ”climate change” (326), ”low carbon” (233), ”forest fire” (180),
”greenhouse gas” (156), and ”clean energy” (144).
\item \textbf{Emergency Management}: The prevalence of ”forest fire” (180) indicates shared cross-border resources for disaster response.
\item \textbf{Regional Partners}: Frequent references to ”British Columbia” (200), ”Great Lakes” (138), and ”Republic [of] China” (124) highlight the primary geopolitical partners: Canada and East Asia.
\end{itemize}

\begin{table}[H]
    \centering
    \caption{Top 10 words and bigrams.}
    \label{tab:top_terms_narrow}
    % \resizebox scales the table to the width of the column (\columnwidth)
    \resizebox{\columnwidth}{!}{%
        \begin{tabular}{|l|r||l|r|}
            \hline
            \textbf{Word} & \textbf{Count} & \textbf{Bigram} & \textbf{Count} \\
            \hline
            state & 921 & memorandum understanding & 904 \\
            parties & 423 & climate change & 326 \\
            information & 422 & economic development & 240 \\
            energy & 380 & low carbon & 233 \\
            la & 262 & confidential information & 214 \\
            understanding & 241 & British Columbia & 200 \\
            supervisors & 233 & forest fire & 180 \\
            participants & 215 & greenhouse gas & 156 \\
            agreement & 162 & clean energy & 144 \\
            member & 138 & Great Lakes & 138 \\
            \hline
        \end{tabular}%
    }
\end{table}

\subsubsection{State-Specific Topics}

State-level analysis highlights distinct regional priorities:

\begin{itemize}
\item \textbf{Western States (Environment)}: California and Oregon are heavily associated with terms regarding ”climate change” and ”clean energy”.
\item \textbf{Texas and Arkansas (Administrative)}: Unlike the environmental focus of the West Coast, these states feature high frequencies of ”driver license” and ”economic development,” indicating agreements focused on commerce, reciprocity, and bureaucracy. \item \textbf{New Jersey (Health Cooperation)}: Unique occurrences of ”Salud Chihuahua” and ”servicios salud” indicate specific health-related cooperation with the Mexican state of Chihuahua, likely reflecting consular or local health initiatives. \item \textbf{Alaska (Indigenous Relations)}: Analysis of Alaska's documents reveals specific references to the ”Nisga Nation,” highlighting unique cross-border cooperation with Indigenous First Nations in British Columbia. \item \textbf{Northern Border States}: Maine, Michigan, Indiana, and Wisconsin frequently feature terms like ”forest fire”, ”crossing agreement”, ”Great Lakes”, and ”Ontario Quebec,” reflecting shared infrastructure and environmental stewardship with Canada.
\end{itemize}

\section{Proof of concept}

\subsection{Tasks 1, 7, 9}

The Proof of Concept (POC) regarding the OCR pipeline establishes the technical feasibility of transforming a large corpus of scanned legal agreements into a structured dataset optimized for computational social science. The proposed system is anchored in a deep learning-based Optical Character Recognition (OCR) framework designed to digitize documents with high geometric fidelity, subsequently enabling complex algorithmic analysis to determine critical metrics including document length, the prevalence of boilerplate language, and the substantive scope of international cooperation.

\subsubsection{Determining documents length}

The core extraction layer is built upon the docTR (Document Text Recognition) library, utilizing a pretrained predictor to process PDF documents ingested from the project’s directory structure. To ensure the integrity of the analysis, the workflow incorporates a preprocessing step that programmatically discards the first page of every file, effectively stripping away the administrative cover sheets and metadata often appended during the download process. Unlike simple text-dumping tools, the model generates a hierarchical JSON output for each agreement, mapping the document’s geometry into a nested structure of pages, blocks, lines, and words. This geometric preservation is crucial for the subsequent metadata extraction task, where a custom algorithm traverses the JSON hierarchy to calculate document volume. By iterating through every text block and summing the token counts at the line level, the system produces a highly accurate word count that ignores whitespace anomalies, resulting in the dataset which facilitates precise quantitative comparisons of agreements (Task 7).

\subsubsection{Analyzing frequency of recurring clauses}

Building upon this digitized foundation, the module implements a sophisticated NLP architecture to analyze the legal text. For the analysis of recurring clauses (Task 9), the system distinguishes between standardized “boilerplate” templates and bespoke diplomatic terms by leveraging the visual block structure detected by the OCR engine. To filter out noise such as page numbers, headers, and artifacts, the system discards any text blocks containing fewer than ten words. The remaining substantive clauses are encoded into dense vector embeddings using the all-MiniLM-L6-v2 Sentence Transformer, a model optimized for semantic similarity tasks. These embeddings are then subjected to HDBSCAN clustering with a Euclidean metric and a minimum cluster size of two, allowing the system to detect even minimal repetitions of legal phrasing. The classification logic is derived from these cluster assignments: clauses are classified to groups appearing more than 10, 5, and 1 time, those more than 5 times could be perceived as standard diplomatic protocol (within one state), whereas clauses that fail to form clusters (labeled as -1) are identified as unique, custom-drafted terms specific to a particular negotiation.

\subsubsection{Identifying areas of cooperation}

In parallel with the structural analysis, the POC addresses the identification of cooperation areas (Task 1) through a Zero-Shot Classification approach powered by the facebook/bart-large-mnli model. This methodology allows the system to categorize agreements into specific sectors—such as “Green Energy,” “Culture \& Arts,” or “Trade \& Economic Development”without the prohibitive requirement of a manually labeled training dataset. Recognizing the input constraints of Transformer-based models, the pipeline reconstructs the full text from the JSON output and applies a truncation strategy, limiting the input to the first 3,000 characters. This window is strategically selected to encompass the Preamble and the initial “Scope of Cooperation” articles, where the binding intent of the agreement is typically articulated. The model performs a multi-label classification on this truncated text, assigning tags only when the confidence score exceeds a 0.5 threshold.

\subsection{Tasks 2, 3, 5}

This section details the POC implementation for extracting agreements parties, agreement types, and identifying international organizations mantioned from the legal agreements corpus. The POC has been executed on the Alabama subdirectory of the parsed International Agreements Database.

\subsubsection{Task 2: Extraction of Agreement Parties}

Two distinct approaches were evaluated for the extraction of contracting parties from the legal texts: a Named Entity Recognition (NER) approach and a Large Language Model (LLM) prompting approach.

Method 1: Legal NER The initial attempt utilized a BERT-based model fine-tuned for legal Named Entity Recognition. The extraction logic involved tokenizing the input text, running inference to obtain labels, and aggregating tokens into entities based on the tagging scheme.

The NER approach proved insufficient for this specific use case. The primary issues observed included:

\begin{itemize}
\item Complex Naming Structures: Agreement parties often possess long, multi-word names containing embedded entities (e.g., ”Ministry of Economics, Small and Medium Business, and Technology”). The NER model frequently fragmented these into separate entities rather than recognizing the single legal party.
\item Model Availability: Finding robust, publicly available NER models fine-tuned specifically for this type of legal document construction was difficult, with several candidate sources being broken or deprecated.
\end{itemize}

Method 2: Generative extraction (Mistral 7B Instruct) Due to the limitations of the NER approach, the strategy shifted to using a quantized version of the Mistral 7B Instruct model. The model was prompted to act as a legal AI and return a JSON object containing the required information. The instruct-oriented fine-tuning of the Instruct varient of the Mistral model was supposed to ensure specifity of the answer. It could be interesting to compare if a model fine-tuned for legal purposes would do better in extraction than a model trained to consider exact instructions.

The LLM approach yielded significantly better results. The model successfully identified complex party names and provided context regarding their roles. However, minor errors persisted in complex scenarios, such as occasionally misclassifying constituent members of a party as separate agreement parties.

\subsubsection{Task 3: Agreement Type Classification}

Two methods were implemented to categorize the agreements into types such as Memorandum of Understanding, Trade Agreement, or Partnership Agreement.

Method 1: Zero-Shot Classification (BART) A BART-large model trained on the MNLI dataset was employed for zero-shot classification. The model classified texts against a predefined list of 13 agreement types.

It was difficult to assess the outcome validity without an expert knowledge, but some of the agreements like Memoranda of Understanding had this value written in the very first line, and for those, the classification worked. Some of the classified agreements had their confidence ratios for each type very close to each other, indicating that the predefined subset could not be accurate.

Method 2: Generative Extraction (Mistral 7B Instruct) To allow for more flexibility, the Mistral 7B model was prompted to analyze the text and generate the agreement type and a brief description without being restricted to a fixed list. This approach allowed for the identification of nuanced agreement types that might not fit strictly into pre-set categories.

Some of the agreements had their extracted type matching for both methods. Generative extraction yields better responses in the case of custom agreements – the exact type of the agreement may not be stated explicitly.

\subsubsection{Task 5: Extraction of International Organizations}

Building on the findings from Task 2, the extraction of international organizations was performed using the Mistral 7B Instruct model rather than NER.

Methodology The model was instructed to extract organizations operating across national borders (e.g., intergovernmental bodies) while explicitly ignoring purely domestic agencies unless they were part of an international body. The output was structured as a JSON object containing the organization’s name and a short context.

Results This method successfully filtered out local entities to focus on international actors, providing a cleaner list of relevant organizations involved in the agreements.

\subsection{Tasks 6: Determine Terms of Validity}

The following methods are implemented to determine the terms of validity

\subsubsection{HeidelTime}
This method applies the HeidelTime temporal tagger to identify \texttt{TIMEX3} \texttt{DATE} and \texttt{DURATION} expressions in agreement text. The resulting TimeML output is parsed and temporally normalized into ISO-formatted dates. Validity-specific anchor heuristics are applied at the sentence level to associate temporal expressions with agreement validity semantics (e.g., \emph{effective}, \emph{enter into force} for start dates; \emph{until}, \emph{expires}, \emph{expiration} for end dates). When explicit anchors are absent, fallback logic assigns the earliest detected date as the effective date and the latest as the end date. Although HeidelTime was evaluated both with and without TreeTagger-based POS tagging, the results reported here correspond to the POS-free configuration, which proved more robust on OCR-derived text and showed no consistent improvement from POS tagging.

To assess the computational overhead of temporal validity extraction, we measured wall-clock runtime and process-level memory usage for the HeidelTime-based pipeline on the evaluation corpus. The results compare configurations with and without TreeTagger-based POS annotation, highlighting the modest runtime cost of POS tagging and stable memory behavior.

\begin{center}
\small
\begin{tabular}{lrrr}
\toprule
\textbf{HeidelTime config} & \textbf{Total (s)} & \textbf{s/doc} & \textbf{RSS $\Delta$ (MB)} \\
\midrule
POS=NO     & 135.05 & 4.50 & -28.68 \\
TreeTagger & 144.06 & 4.80 & +0.03 \\
\bottomrule
\end{tabular}
\captionof{table}{Runtime and memory for HeidelTime-based validity extraction on 30 documents.}
\label{tab:runtime_memory}
\end{center}



\subsubsection{POC Main Validity Output (Hybrid Validity Extractor)}
The POC main method implements a hybrid extraction pipeline combining rule-based retrieval, neural semantic filtering, and heuristic temporal parsing. Validity candidate sentences are first retrieved using keyword triggers and context expansion. These candidates are then filtered using zero-shot classification and NLI-style verification to retain only sentences that semantically entail a validity clause. From the verified evidence, dates and durations are extracted using rule-based temporal parsing and assigned to effective date, end date, or duration using anchor-based heuristics. The method outputs a document-level validity status together with supporting evidence spans.

\subsubsection{POC Baseline Keyword Validity (Presence-Only Flag)}
This baseline detects the presence of validity language without performing any temporal extraction. The document text is scanned for predefined validity-related keywords using regular expressions. If any keyword is found, the document is marked as containing a validity clause. This method does not extract or normalize dates or durations and serves as a minimal reference point for evaluating the benefit of structured temporal extraction.

\subsubsection{POC Baseline Rule-Based Validity (Simple Patterns)}
This approach relies exclusively on hand-crafted regular expressions to extract validity-related temporal information. Patterns are used to identify effective dates, end dates, and durations directly from the text. Extracted temporal expressions are normalized where possible, and simple positional heuristics are applied to distinguish start and end dates. Unlike the hybrid method, this baseline does not employ semantic filtering or neural verification, making it computationally lightweight but sensitive to phrasing variation and OCR noise.

The results obtained by above discussed techniques are as following:

\begin{center}
\captionsetup{type=table}
\small
\setlength{\tabcolsep}{4pt}
\renewcommand{\arraystretch}{1.15}

\begin{tabular}{p{3.6cm}rrrr}
\toprule
\textbf{Technique} & \textbf{N} & \textbf{Found} & \textbf{Start} & \textbf{End} \\
\midrule
HeidelTime  & 30 & 100.0 & 100.0 & 83.3 \\
POC main (hybrid)           & 30 & 76.7  & 46.7  & 16.7 \\
POC keyword (presence)      & 30 & 80.0  & 0.0   & 0.0  \\
POC rules (patterns)        & 30 & 80.0  & 53.3  & 10.0 \\
\bottomrule
\end{tabular}

\caption{Term validity extraction coverage (\%) on 30 California agreements.}
\label{tab:validity-coverage}
\end{center}
\begin{center}
\captionsetup{type=table}
\small
\setlength{\tabcolsep}{4pt}
\renewcommand{\arraystretch}{1.15}

\begin{tabular}{p{3.6cm}rrr}
\toprule
\textbf{Technique} & \textbf{N} & \textbf{Dur.} & \textbf{End (explicit)} \\
\midrule
HeidelTime  & 30 & 83.3 & 13.3 \\
POC main (hybrid)           & 30 & 70.0 & 16.7 \\
POC keyword (presence)      & 30 & 0.0  & 0.0  \\
POC rules (patterns)        & 30 & 70.0 & 10.0 \\
\bottomrule
\end{tabular}

\caption{Duration extraction and explicitly stated end dates (\%) on 30 California agreements.}
\label{tab:validity-duration}
\end{center}


\FloatBarrier

The above results show that HeidelTime, combined with validity-specific anchor heuristics provides the most reliable extraction of structured validity information, achieving the highest coverage for effective dates, end dates, and durations. Simpler baseline approaches are sufficient for detecting the presence of validity clauses but fail to consistently recover explicit temporal boundaries, highlighting the importance of structured temporal normalization.

\begin{table}[H]
\centering
\small
\resizebox{\columnwidth}{!}{%
\begin{tabular}{lcccccc}
\toprule
\textbf{State} & \textbf{\#Docs} & \textbf{Wall (s)} & \textbf{CPU (s)} & \textbf{RSS Peak (MB)} & \textbf{RSS $\Delta$ (MB)} & \textbf{Py Heap (MB)} \\
\midrule
California & 30 & 137.39 & 948.31 & 840.84 & 1.76 & 0.59 \\
\bottomrule
\end{tabular}
}
\caption{Runtime and memory usage for Tasks 8 and 11 on 30 international agreements from California. Results correspond to a single end-to-end execution including lexical retrieval, zero-shot MNLI classification, NLI-style verification, and rule-based post-processing.}
\label{tab:profiling_tasks8_11}
\end{table}
\FloatBarrier



\subsection{Task 8: Determine Conditions for Extending the Agreement}

The proof-of-concept for identifying conditions for extending or renewing an agreement follows a hybrid clause-detection pipeline inspired by state-of-the-art contract analysis approaches evaluated on datasets such as CUAD and LEDGAR. In the first stage, high-recall lexical rules are applied to OCR-extracted agreement text to retrieve candidate clauses containing renewal- and extension-related triggers, with additional context expansion to account for OCR noise and clause fragmentation.

In the second stage, candidate clauses are subjected to zero-shot clause-type classification using an MNLI-based transformer model, serving as a lightweight proxy for task-specific legal language models without requiring fine-tuning. To further reduce false positives, an NLI-style verification step is applied to assess whether the candidate text semantically entails the presence of renewal or extension conditions. Finally, rule-based post-processing distinguishes between automatic renewals and extensions requiring mutual agreement, and extracts associated notice periods where explicitly stated. Supporting evidence spans are retained for interpretability and manual validation.

Only 40 percent of the analyzed agreements contain explicit evaluation or review clauses, confirming that evaluation of implementation is optional and far less standardized than validity or renewal provisions.

\begin{center}
\captionsetup{type=table}
\small
\setlength{\tabcolsep}{6pt}
\renewcommand{\arraystretch}{1.15}

\begin{tabular}{lrr}
\toprule
\textbf{Evaluation status} & \textbf{Count} & \textbf{Percent (\%)} \\
\midrule
Evaluation clause found & 12 & 40.0 \\
Evaluation clause absent & 18 & 60.0 \\
\bottomrule
\end{tabular}

\caption{Presence of the clauses determining the condition of extension across 30 California agreements}
\label{tab:task18-baseline}
\end{center}


\subsubsection{Task 11: Indicate Whether the Agreement Includes an Evaluation of Its Implementation}

The proof-of-concept for detecting clauses related to the evaluation of agreement implementation is implemented using a hybrid retrieval and semantic verification approach informed by prior work on legal clause classification. Initially, high-recall lexical patterns are used to identify candidate text segments referencing evaluation, review, monitoring, or assessment of an agreement’s implementation, ensuring robustness to OCR-induced variability.

Candidate segments are then processed using a zero-shot MNLI-based classifier to assess their relevance to evaluation semantics. An additional NLI-style verification step determines whether the candidate text entails an explicit obligation or mechanism for evaluating implementation, analogous to ContractNLI-style reasoning but without task-specific fine-tuning. The final output is a binary indicator denoting the presence or absence of implementation evaluation clauses, accompanied by extracted evidence spans to support transparency and downstream analysis.

\begin{center}
\captionsetup{type=table}
\small
\setlength{\tabcolsep}{6pt}
\renewcommand{\arraystretch}{1.15}

\begin{tabular}{lrr}
\toprule
\textbf{Renewal type} & \textbf{Count} & \textbf{Percent (\%)} \\
\midrule
By mutual agreement & 19 & 63.3 \\
Automatic renewal & 6 & 20.0 \\
No renewal clause (absent) & 5 & 16.7 \\
\bottomrule
\end{tabular}

\caption{Task 11: Distribution of renewal clauses across 30 California agreements.}
\label{tab:task11-renewal}
\end{center}

Most agreements specify renewal by mutual agreement, while automatic renewal is comparatively rare and a non-negligible fraction of agreements contain no explicit renewal provision.


\subsection{Tasks 4, 10, 12, 13}

\subsubsection{Introduction}

The objective of this Proof of Concept (POC) is to develop a system for the automated analysis of legal
agreements. The project focuses on leveraging State-of-the-Art (SOTA) Natural Language Processing
(NLP) techniques to handle deduplication, semantic clustering, and metadata normalization for a
dataset of approximately 30 legal documents.

\subsubsection{Methodology and SOTA Tasks}

Semantic Embeddings To capture the nuances of legal language, we utilized Legal-SBERT (based
on paraphrase-multilingual-MiniLM-L12-v2 ). This model transforms raw text into high-dimensional
dense vectors, preserving semantic relationships between documents.

Deduplication (SOTA) We implemented a deduplication pipeline using:

\begin{itemize}
\item \textbf{Cosine Similarity Matrix}: To compute pair-wise similarity between all documents.
\item \textbf{FAISS (Facebook AI Similarity Search)}: For efficient indexing and retrieval of near-
duplicate documents.
\end{itemize}

A similarity threshold of 0.85 was applied to identify redundant agreements.

Semantic Similarity Clustering For document discovery and categorization, we combined:

\begin{itemize}
\item \textbf{UMAP (Uniform Manifold Approximation and Projection)}: To reduce dimensionality
while preserving global structure.
\item \textbf{HDBSCAN (Hierarchical Density-Based Spatial Clustering of Applications with
Noise)}: To automatically detect clusters of agreements based on their content without pre-
defining the number of clusters.
\end{itemize}


\begin{table}[H]
\centering
\small
\caption{Top High-Similarity Document Pairs Identified (Deduplication Candidates)}
\label{tab:deduplication}
\begin{tabular}{@{}llc@{}}
\toprule
\textbf{Document A} & \textbf{Document B} & \textbf{Similarity Score} \\
\midrule
Arkansas/Arkansas\_6.txt & Arkansas/Arkansas\_8.txt & \textbf{1.000} \\
Arkansas/Arkansas\_9.txt & California/1March15.txt & 0.997 \\
Arkansas/Arkansas\_4.txt & Arkansas/Arkansas\_5.txt & 0.987 \\
Texas/Texas\_30.txt & Texas/Texas\_17.txt & 0.981 \\
California/14182022.txt & California/1October2.txt & 0.975 \\
Utah/Utah\_13.txt & California/1August23.txt & 0.974 \\
\bottomrule
\end{tabular}
\end{table}
% ==========================================================================================

Metadata Normalization \& Consistency Check We addressed the inconsistency in extracted
data (e.g., ”NYC” vs ”New York City”) using:

\begin{itemize}
\item \textbf{RapidFuzz}: Fuzzy matching to merge variations of the same entity.
\item \textbf{LLM-Assisted Normalization}: Standardizing noisy fields (dates, organization names) into a
uniform format (e.g., YYYY-MM-DD).
\item \textbf{Consistency Checks}: Rule-based verification to ensure the extracted metadata is reliable and
present in the source text.
\end{itemize}

\subsubsection{POC Results and Project Tasks}

The pipeline was executed on a stratified sample of 30 agreements (incorporating Arkansas-specific documents). Specific results for the targeted tasks are detailed below:

\begin{table}[H]
\centering
\small
\setlength{\tabcolsep}{6pt}
\begin{tabular}{@{}p{1.2cm} p{\dimexpr\columnwidth-1.2cm\relax}@{}}
\toprule
\textbf{Task ID} & \textbf{Description and Result} \\
\midrule
(4) & Sister Cities International \%: The improved regex search (handling variants like ``SCI'', ``Sister City'') identified that \textbf{10.00\%} of the stratified sample fall under this category. \\
(10) & Detailed Agreements: Partner analysis revealed significant variance in verbosity. The most detailed agreements were associated with headers like ``SEAL OF'' (Avg: 3,967 words) and ``Multilateral Memorandum'' (Avg: 3,920 words), compared to standard Memoranda (approx. 1,300 words). \\
(12) & Coordination Detection: The boolean flag system successfully parses coordination clauses. In validation runs, documents such as \texttt{Utah/Utah\_18.txt} were correctly flagged as \texttt{True} for containing phrases like ``coordination with''. \\
(13) & Legal References: Automated detection of citations was highly effective. The system flagged agreements including \texttt{California/14182022.txt} and \texttt{California/1October2.txt} as \texttt{True} for referencing external statutes (e.g., ``pursuant to'', ``see''). \\
\bottomrule
\end{tabular}
\caption{Fulfillment of Specific Project Tasks with Quantitative Findings}
\label{tab:tasks}
\end{table}


\section{Final Results}

\subsection{Task 1: Identification of Areas of Cooperation}

To identify the thematic scope of the agreements, we employed a multi-label zero-shot classification approach. Unlike traditional supervised learning, which requires annotated training data for every specific label, zero-shot classification allows the model to predict class relevance based on descriptive labels alone.

We utilized the \texttt{facebook/bart-large-mnli} model, a pre-trained textual entailment model capable of determining the probability that a given text sequence corresponds to a specific topic label. We defined a taxonomy of potential cooperation areas, including ``Technology \& Innovation,'' ``Environment \& Green Energy,'' ``Trade \& Economic Development,'' and ``Education \& Students.''

For each agreement, the full text was passed to the model, which assigned a confidence score to each label. A threshold of 0.5 was applied; labels exceeding this confidence score were considered ``active topics'' for the document. This multi-label configuration acknowledges that a single international agreement often covers multiple overlapping sectors (e.g., a trade agreement that also includes educational exchange provisions).

The analysis revealed a strong emphasis on modernization and sustainability within state-level international agreements. As illustrated in Figure~\ref{fig:topic_dist}, ``Technology \& Innovation'' and ``Environment \& Green Energy'' emerged as the most frequent areas of cooperation, surpassing traditional categories such as ``Culture \& Arts.''

\begin{figure}[H]
    \centering
    \includegraphics[width=\linewidth]{images/top_topic_dist.png}
    \caption{Frequency of all detected topics across the dataset. Technology and Environmental cooperation are the dominant themes.}
    \label{fig:topic_dist}
\end{figure}

Geographically, the distribution of topics varies significantly by state. Figure~\ref{fig:state_topics} highlights that larger economies like California and Texas exhibit the highest volume of agreements in ``Trade \& Economic Development'' and ``Technology.'' Conversely, states with fewer agreements often focus on specific niche partnerships. The heatmap analysis confirms that while some topics are universally relevant, the intensity of cooperation in high-tech and green energy sectors is concentrated in key states with established international policy frameworks.

The time and memory consumption for this task is described in Table \ref{tab:runtime_memory_bart_large_mnli}.

\begin{center}
\small
\begin{tabular}{lr}
\toprule
\textbf{Metric} & \textbf{Value} \\
\midrule
Number of documents & 298 \\
Total runtime (s) & 410 \\
Average time per document (s) & 1.38 \\
Peak Python memory (GB) & 4.7 \\
\bottomrule
\end{tabular}
\captionof{table}{Runtime and memory profile for facebook/bart-large-mnli on the full dataset.}
\label{tab:runtime_memory_bart_large_mnli}
\end{center}

\begin{figure}[H]
    \centering
    \includegraphics[width=\linewidth]{images/topic_state_heatmap_sorted.png}
    \caption{Heatmap of cooperation areas by state, showing the density of agreements in specific sectors.}
    \label{fig:state_topics}
\end{figure}

\subsection{Task 2 - Agreement Parties}

\subsubsection{Task 2 methodology}

For Task 2, we extracted the contracting parties of each agreement. We first implemented a baseline using a legal NER model (dandoune/legal-NER) as a token classification approach. We ran the model on a representative sample document per state, converted predicted token labels into spans, and saved the resulting entity fragments. We used this primarily as a sanity check, because we observed that token-level NER often breaks long party names into fragments and struggles when party strings include nested entities such as countries, agencies, and individual names.

We then implemented our main approach using an instruction-tuned LLM (Mistral-7B Instruct, loaded from a GGUF file via llama-cpp-python). We prompted the model to extract only the direct contracting parties and to return a strict JSON object containing a list of parties with name and role. We stored both the raw model output and a parsed JSON version when parsing succeeded. We measured extraction quality at the system level by tracking JSON parse success.

\subsubsection{Task 2 results discussion}

\begin{figure}[H]
    \centering
    \includegraphics[width=0.3\linewidth]{images/ner.png}
    \caption{Sample NER output.}
    \label{fig:ner}
\end{figure}

The output on Figure \ref{fig:ner} shows that the NER model captures isolated tokens and partial spans, but it does not reliably assemble the complete contracting-party strings that are typical in agreements (which often contain multiple nested entities, long titles, and jurisdiction descriptors). Because the extracted entities are fragmented and the label IDs do not map cleanly to party-specific categories, the approach does not support accurate party extraction

Mistral prompting method worked great for the task goal because it produced structured party lists that can be aggregated across documents. JSON parsing succeeded for almost all outputs (≈ 98\% parse success across processed documents), which made the results usable in a real-world scenario.

\begin{figure}[H]
    \centering
    \includegraphics[width=\linewidth]{images/n_parties.png}
    \caption{Number of parties extracted histogram.}
    \label{fig:n_parties}
\end{figure}

Mistral usually extracted 2 parties only (Figure \ref{fig:n_parties}i), which is an expected outcome.

\subsubsection{Task 2 time and memory efficiency}

\begin{table}[ht]
\centering
\caption{NER profiling summary}
\label{tab:profile_task2_ner_single}
\begin{tabular}{ll}
\hline
Metric & Value \\
\hline
Max RSS (MB) & 5736.453 \\
Mean time (s) & 1.282 \\
\hline
\end{tabular}
\end{table}

\begin{table}[ht]
\centering
\caption{Mistral inference profiling summary}
\label{tab:profile_task2_parties_mistral}
\begin{tabular}{ll}
\hline
Metric & Value \\
\hline
Max RSS (MB) & 5833.953 \\
Mean time (s) & 304.395 \\
\hline
\end{tabular}
\end{table}

\subsection{Task 3 - Agreement Type}

\subsubsection{Task 3 methodology}

For Task 3, we identified the agreement type of each document using two complementary methods. We first applied zero-shot classification using facebook/bart-large-mnli through the HuggingFace pipeline. We defined a fixed label set of common agreement categories and we classified each document into exactly one label. We saved the predicted label and its confidence score, and we also kept the top-3 candidates for later comparison and reporting. We used this approach because it produces consistent categories suitable for aggregate plots and tables, while still providing a confidence measure that can be used for triage.

We then performed open-ended agreement type extraction with Mistral. We prompted the model to return a JSON object containing agreement type and a short description. We stored both the raw output and the parsed JSON when possible, and we tracked parsing success to quantify how reliably the model produced structured responses. We used this method because it is not constrained by a fixed label set and can capture types that were not included in the zero-shot taxonomy, at the cost of potentially more variability in naming and normalization.

\subsubsection{Task 3 results discussion}

\begin{figure}[H]
    \centering
    \includegraphics[width=\linewidth]{images/bart_n.png}
    \vspace{1em}
    \includegraphics[width=\linewidth]{images/mistral_n.png}
    \caption{Types of agreements detected by both methods.}
    \label{fig:type_n}
\end{figure}

As visible on Figure \ref{fig:type_n} BART spreads predictions across several labels (MOU, Partnership, Sister State Agreement), suggesting the fixed label set strongly shapes the output. Mistral, on the other hand, strongly collapses most documents into \textit{Memorandum of Understanding}. This can be good if the corpus truly contains mostly MOUs, but it can also indicate a bias toward a common legal template phrase. Zero-shot classification results are much more realistic.

\begin{figure}[H]
    \centering
    \includegraphics[width=\linewidth]{images/bart_confidence.png}
    \caption{BART classification confidence.}
    \label{fig:conf}
\end{figure}

Confidence from Figure \ref{fig:conf} is moderate overall (many scores ~0.2–0.6), which is expected for zero-shot on OCR text. The higher-confidence predictions tend to be for common classes (MOU/Partnership). Rare classes are less stable.

\subsubsection{Task 3 time and memory efficiency}

\begin{table}[ht]
\centering
\caption{BART zero-shot classification profiling summary}
\label{tab:profile_task3_type_bart}
\begin{tabular}{ll}
\hline
Metric & Value \\
\hline
Max RSS (MB) & 7488.699 \\
Mean time (s) & 186.082 \\
\hline
\end{tabular}
\end{table}

\subsection{Task 5 - International Organizations Mentioned}

\subsubsection{Task 5 methodology}

For Task 5, we extracted international organizations mentioned in the agreements using the Mistral model. We initially considered reusing an NER-based approach similar to Task 2, but we did not proceed with NER for this task because the Task 2 NER baseline produced fragmented and unreliable entity spans on this OCR-heavy legal text (especially for long, compound entity names). Based on that outcome, we chose to rely directly on LLM-based extraction for Task 5.

We prompted Mistral to output a JSON object containing a list of international organizations with name and a short type description, and we explicitly instructed the model to focus on cross-border or intergovernmental bodies while ignoring purely domestic entities unless they were clearly part of an international structure. We stored both the raw model output and the parsed JSON (when parsing succeeded) and tracked JSON parse success as a basic reliability metric.

\subsubsection{Task 5 results discussion}
Mistral produced, similarly to Task 2, parseable structured output with a high success rate (≈ 92\% parse success rate). Most analyzed documents did not mention any international organizations according to model's predictions (Figure \ref{fig:n_orgs}). The main observed limitation was semantic: \textit{international organization} boundaries are sometimes ambiguous in analyzed texts, and the model can occasionally include national ministries or governments depending on context. Still, the results are strong enough for descriptive analysis.

\begin{figure}[H]
    \centering
    \includegraphics[width=\linewidth]{images/n_orgs.png}
    \caption{Number of extracted international organizations histogram.}
    \label{fig:n_orgs}
\end{figure}


\subsubsection{Task 5 time and memory efficiency}

Time and memory efficiency for Task 5 align with these reported for Task 2 as the same method was used to extract the information.

\subsection{Runtime and Memory Profiling for task 8 and 11}

\begin{table}[H]
\centering
\small
\begin{tabular}{l r}
\toprule
\textbf{Metric} & \textbf{Value} \\
\midrule
Number of agreements & 298 \\
Total wall-clock time (s) & 1423.41 \\
Total CPU time (s) & 9511.17 \\
Average time per agreement (s) & 4.78 \\
RSS memory at start (MB) & 755.17 \\
RSS memory at end (MB) & 774.96 \\
Peak RSS memory (MB) & 779.70 \\
RSS memory increase (MB) & 19.79 \\
Peak Python heap (MB) & 2.83 \\
\bottomrule
\end{tabular}
\caption{Runtime and memory profiling for the full dataset evaluation of Tasks~8 and~11 across all U.S. state agreements. The entire dataset was processed in a single pass without re-execution for profiling.}
\label{tab:runtime_memory_tasks8_11}
\end{table}

We evaluated the complete dataset of 298 U.S. state-level agreements for Tasks~8 and~11 in a single end-to-end run, ensuring that runtime and memory measurements were collected without requiring repeated executions.

These results demonstrate that the proposed pipeline scales reliably to several hundred long-form legal agreements and can be executed efficiently in a single-pass setting, making it suitable for large-scale agreement mining without repeated recomputation for performance analysis.

\subsection{Task 6: Term of Validity}
The following table reports end-to-end runtime and memory consumption for TreeTagger-based HeidelTime validity extraction on the full evaluation corpus.

\begin{center}
\small
\begin{tabular}{lr}
\toprule
\textbf{Metric} & \textbf{Value} \\
\midrule
Number of documents & 298 \\
Documents with errors & 3 \\
Total runtime (s) & 813.66 \\
Average time per document (s) & 2.71 \\
Median time per document (s) & 2.55 \\
95th percentile time per document (s) & 3.71 \\
Peak Python memory (MB) & 2.16 \\
\bottomrule
\end{tabular}
\captionof{table}{Runtime and memory profile for TreeTagger-based HeidelTime validity extraction on the full dataset.}
\label{tab:full_runtime_memory}
\end{center}

Table 12 summarizes the computational cost of running the TreeTagger-based HeidelTime validity extraction pipeline on the full dataset of 298 agreements. All documents are processed regardless of extraction outcome, and runtime and memory usage reflect end-to-end temporal normalization and clause-level analysis.

\begin{center}
\small
\begin{tabular}{lr}
\toprule
\textbf{End date source} & \textbf{Agreements} \\
\midrule
Explicitly stated & 28 \\
Implicit / inferred & 161 \\
Not available & 109 \\
\bottomrule
\end{tabular}
\captionof{table}{Source of extracted end dates across agreements. Explicit end dates are directly stated in the text, while implicit end dates are inferred from temporal context.}
\label{tab:end_date_source}
\end{center}

Table 13 reports the availability and source of agreement end dates identified by the system. While all agreements are processed, only a subset explicitly states an end date, with additional cases inferred via heuristic fallback rules, and the remainder lacking sufficient temporal information.

\begin{center}
\small
\setlength{\tabcolsep}{3pt}
\captionsetup{type=table}
\caption{Extracted agreement validity periods with document evidence.}
\label{tab:task6_top_examples_verified}

\begin{tabularx}{\columnwidth}{p{0.25\columnwidth} p{0.14\columnwidth} p{0.14\columnwidth} p{0.14\columnwidth} X}
\toprule
\textbf{Document} & \textbf{Start Date} & \textbf{End Date} & \textbf{Duration} & \textbf{Extracted Evidence} \\
\midrule

Texas\_48 & 2019-09-27 & 2021-08-31 & -- &
``This Statement of Mutual Cooperation is intended to be in place until August 31, 2021.'' \\

Texas\_50 & 2023-04-27 & 2025-03-31 & -- &
``DURATION OF OPERATION ... applied from April 27, 2023, until March 31, 2025 ...'' \\

NewJersey\_9 & 2008-05-07 & 2010-09-30 & -- &
``... will terminate on September 30, 2010 ...'' \\

Alabama\_3 & -- & -- & three years &
``This Memorandum of Intent shall enter into effect upon signature and shall remain in force for a period of three years.'' \\


Arkansas\_6 & 2018-01-01 & -- & five years &
``This Agreement shall be in full force and effective for a period of 5 years.'' \\

\bottomrule
\end{tabularx}
\end{center}

Table 14 presents representative high-confidence agreement validity periods extracted by the system, including start dates, end dates, or explicitly stated durations, along with the supporting textual evidence. These examples illustrate cases where temporal validity is expressed using clear legal force language, such as effective dates, termination clauses, or fixed-term durations.



\subsection{Task 8: Renewal and Extension}

\begin{table}[H]
\centering
\caption{Distribution of renewal and extension clause types across $N=298$ agreements.}
\label{tab:task8_renewal_distribution}
\begin{tabular}{lrr}
\toprule
Renewal type & Count & Percent (\%) \\
\midrule
By mutual agreement & 102 & 34.23 \\
No renewal clause   & 101 & 33.89 \\
Automatic renewal   & 85  & 28.52 \\
Uncertain / ambiguous & 10 & 3.36 \\
\bottomrule
\end{tabular}
\end{table}

Table 12 shows that renewal provisions are distributed relatively evenly across agreement types. Slightly over one third of the agreements require explicit mutual agreement for renewal, while a comparable proportion contains no renewal clause at all. Automatic renewal is present in just under one third of the documents, indicating that fixed-term agreements with implicit continuation are common. Only a small fraction of cases remain uncertain, typically due to vague or underspecified language that prevents a confident classification.

\begin{center}
\small
\setlength{\tabcolsep}{3pt}
\captionsetup{type=table}
\caption{Renewal and extension clauses extracted by the system (Task~8).}
\label{tab:task8_examples}

\begin{tabularx}{\columnwidth}{p{0.3\columnwidth} p{0.24\columnwidth} X}
\toprule
\textbf{Document} & \textbf{Renewal Type} & \textbf{Extracted Evidence} \\
\midrule
Alabama\_4 & By mutual agreement &
``Additional items may be added to this document as mutually agreed upon by both parties.'' \\

Alaska\_2 & Automatic &
``Renewed as of September 11, 2015, September 11, 2017, September 11, 2019 and September 11, 2021.'' \\

California\_1July5 & Automatic &
``Unless canceled by any party, this MOU will automatically be renewed for one-year periods.'' \\
\bottomrule
\end{tabularx}
\end{center}



\subsection{Task 7: Quantitative Analysis of Agreement Length}

We performed a quantitative analysis of the physical and lexical length of the agreements to understand the complexity and detail level of these legal documents.

The length of each agreement was measured using two metrics: page count and total word count. Page counts were extracted directly from the PDF metadata and layout analysis, while word counts were derived following Optical Character Recognition (OCR) and text extraction. This dual approach helps account for formatting differences, where font size or layout might skew the perception of length based on page count alone.

The distribution of agreement lengths is highly right-skewed. As shown in Figure~\ref{fig:length_dist}, the vast majority of agreements are concise, typically spanning between 1 to 5 pages. This suggests that most state-level agreements are high-level Memoranda of Understanding (MoUs) rather than detailed, multi-chapter treaties.

\begin{figure}[H]
    \centering
    \includegraphics[width=\linewidth]{images/pages_distribution_barplot.png}
    \vspace{1em}
    \includegraphics[width=\linewidth]{images/words_distribution_barplot.png}
    \caption{Overall distribution of page counts (top) and total words (bottom). The data indicates a prevalence of short, high-level documents.}
    \label{fig:length_dist}
\end{figure}

However, significant outliers exist. Figure~\ref{fig:state_length_boxplot} presents a boxplot analysis by state, revealing that California and Texas not only have the most agreements but also the greatest variance in document length. California, in particular, exhibits several extensive agreements that significantly exceed the median, likely reflecting complex regulatory alignments or multi-jurisdictional frameworks. In contrast, states like Arkansas and Alabama show more uniformity in document length, suggesting a standardized template or scope for their international engagements.

This task didn't require substantial memory or runtime resources.

\begin{figure}[H]
    \centering
    \includegraphics[width=\linewidth]{images/states_page_count_boxplot_solid.png}
    \vspace{1em}
    \includegraphics[width=\linewidth]{images/states_word_count_boxplot_solid.png}
    \caption{Distribution of page (top) and word (bottom) counts by state. Boxplots highlight the variability and presence of complex, lengthy agreements in major states.}
    \label{fig:state_length_boxplot}
\end{figure}


\subsection{Task 9: Analysis of Recurring Clauses}

To assess the standardization of legal language across the corpus, we analyzed the frequency of recurring clauses and common lexical patterns.

We adopted a semantic clustering approach to identify recurring clauses, as exact string matching fails to capture variations in wording that convey the same legal meaning. The text of the agreements was segmented into clause-level blocks. We then generated semantic embeddings for each block using the \texttt{all-MiniLM-L6-v2} model.

These high-dimensional embeddings were clustered using HDBSCAN (Hierarchical Density-Based Spatial Clustering of Applications with Noise). This method allows for the identification of semantically similar groups of text (e.g., liability waivers or termination clauses) without requiring a pre-defined number of clusters.

The lexical analysis (Figure~\ref{fig:wordcloud}) highlights the terminology central to these documents. Unigrams such as ``cooperation,'' ``development,'' and ``energy'' dominate, while bigram analysis reveals specific focus areas like ``clean energy,'' ``economic development,'' and ``climate change.''

\begin{figure}[H]
    \centering
    \includegraphics[width=\linewidth]{images/wordcloud_unigrams.png}
    \vspace{1em}
    \includegraphics[width=\linewidth]{images/wordcloud_bigrams.png}
    \caption{Word cloud of most frequent unigrams (top) and list of top bigrams (bottom), illustrating the core vocabulary of the agreements.}
    \label{fig:wordcloud}
\end{figure}

The structural analysis of clauses indicates a low degree of standardization for boilerplate provisions and a significant uniqueness in substantive sections. Figure~\ref{fig:clause_freq} shows the distribution of clause types by frequency.

\begin{figure}[H]
    \centering
    \includegraphics[width=\linewidth]{images/clause_distribution_granular.png}
    \vspace{1em}
    \includegraphics[width=\linewidth]{images/state_clause_composition_vertical.png}
    \caption{Distribution of clause types by frequency (top) and composition by state (bottom). A large portion of text blocks are unique (Rarely), less portion of them represents highly recurring language.}
    \label{fig:clause_freq}
\end{figure}

A large portion of the text blocks (labeled ''Rarely'') appear only once, representing the specific, custom content of each agreement (e.g., specific project details). Lower number of clauses appear ''More than 10 times'' or ''More than 5 times'', corresponding to standard legal provisions such as entry into force, dispute resolution, and validity terms. This distribution confirms that the core subject matter is tailored to each partnership, only sometimes the content relies on recurring, standardized language.

The time and memory consumption for this task is described in Table \ref{tab:runtime_memory_all_minilm}.

\begin{center}
\small
\begin{tabular}{lr}
\toprule
\textbf{Metric} & \textbf{Value} \\
\midrule
Number of documents & 298 \\
Total runtime (s) & 122 \\
Average time per document (s) & 0.41 \\
Peak Python memory (GB) & 4.7 \\
\bottomrule
\end{tabular}
\captionof{table}{Runtime and memory profile for all-MiniLM-L6-v2 on the full dataset.}
\label{tab:runtime_memory_all_minilm}
\end{center}



\subsection{Task 11: Evaluation of Implementation}
\begin{table}[H]
\centering
\caption{Task 11: Presence of implementation evaluation or review clauses in the agreements across $N=298$ agreements.}
\label{tab:task11_evaluation}
\begin{tabular}{lrr}
\toprule
evaluation\_present & count & percent \\
\midrule
Absent & 192 & 64.430000 \\
Present & 106 & 35.570000 \\
\bottomrule
\end{tabular}
\end{table}

As reported in Table 13, explicit clauses requiring evaluation, monitoring, or review of implementation are present in approximately one third of the agreements. The majority of documents do not include such provisions, suggesting that implementation oversight is not systematically formalized in state-level agreements and is often left implicit or to informal coordination mechanisms.


% -------------------------------------------------------------------------
% TASK 4: Sister Cities
% -------------------------------------------------------------------------
\subsection{Task 4: Determination of Sister Cities International Patronage}

The objective of this task was to quantify the percentage of agreements that explicitly operate under the patronage of ``Sister Cities International'' (SCI). Given the historical significance of SCI in US state diplomacy, identifying these agreements distinguishes standard diplomatic memorandums from community-driven exchange programs.

\subsubsection{Methodology}
We applied a baseline exact match searched for the full string ``Sister Cities International''. An optimized SOTA regex solution was then implemented to account for common abbreviations and variations, such as ``SCI'' or ``Sister Cities Int.''.

\subsubsection{Results}
The analysis indicates that SCI agreements constitute a small minority of the state-level dataset:
\begin{itemize}
    \item \textbf{Exact Match (Baseline):} Only \textbf{1.01\%} (3 agreements) contained the full unmodified string.
    \item \textbf{Regex Match (Optimized):} The detection rate rose to \textbf{1.68\%} (5 agreements) when accounting for variations.
\end{itemize}


% -------------------------------------------------------------------------
% TASK 10: Depth Analysis
% -------------------------------------------------------------------------
\subsection{Task 10: Analysis of Agreement Depth and Detail}

The purpose of this task was to analyze the depth of the agreements based on word counts to identify which partners provide the most detailed documentation
Results:
\begin{itemize}
    \item \textbf{High Complexity}: The most detailed "documents" are actually Appendices (e.g., "Appendix I to the MOU"), which often exceed 7,000 words. This suggests that the core technical details are stored in attachments rather than the main agreement text.
    \item \textbf{Institutional Level}: Agreements signed by State Governors or the State of Alaska generally range between 700 and 800 words.
    \item \textbf{Insight}: State-level agreements are more comprehensive and "wordy" than local city-to-city memoranda, reflecting a higher level of formalization.
\end{itemize}

\begin{figure}[H]
    \centering
    \includegraphics[width=\linewidth]{images/Zrzut ekranu 2026-01-25 o 23.57.32.png}
    {TOP 10 partners by average agreement length}
    \label{fig:clause_freq}
\end{figure}

We also categorized agreements into seven semantic clusters (e.g., ``Technology, Science \& Innovation'', ``Economic, Trade \& Investment'') based on our HDBSCAN clustering examination to contextualize the verbosity.

]\begin{table}[H]
\centering
\small
\caption{Taxonomy of of Potential Cooperation Areas and Semantic Keywords}
\label{tab:agreement_categories}
\renewcommand{\arraystretch}{1.3} % Adds a bit of padding between rows
\begin{tabularx}{\linewidth}{@{} >{\bfseries}p{3.5cm} X @{}}
\toprule
\textbf{Category} & \textbf{Keywords \& Semantic Scope} \\
\midrule
Sister Cities \& Cultural Exchange & 
sister cities, cultural exchange, people to people, arts, film, language, community programs, youth exchange, twinning \\
\midrule
Economic, Trade \& Investment & 
trade, investment, economic development, business cooperation, market access, small and medium enterprises, commerce, industry \\
\midrule
Environmental \& Climate Action & 
climate change, forests, emissions, sustainability, environmental protection, great lakes, renewable energy, water management, conservation \\
\midrule
Administrative, Legal \& Regulatory & 
confidentiality, supervision, legal framework, administrative cooperation, information sharing, mmou, regulatory affairs, enforcement \\
\midrule
Education \& Academic Research & 
universities, students, academic exchange, research collaboration, higher education, scholarship, vocational training \\
\midrule
Technology, Science \& Innovation & 
innovation, biotechnology, science and technology, startup, digital economy, research and development, r\&d \\
\midrule
Public Safety \& Emergency Management & 
disaster relief, law enforcement, public health, emergency response, security cooperation \\
\bottomrule
\end{tabularx}
\end{table}


\begin{figure}[H]
    \centering
    \includegraphics[width=\linewidth]{images/TAB.png}
    {Final results of document categorization}
    \label{fig:tab}
\end{figure}

% -------------------------------------------------------------------------
% TASKS 12 & 13: Coordination & Legal References
% -------------------------------------------------------------------------
\subsection{Tasks 12 \& 13: Operational Flags (Coordination and Legal References)}

We implemented boolean flag detection to assess the operational validity of the agreements. Specifically, we sought to determine if the agreements establish coordination with third parties (Task 12) and if they are anchored in existing legal statutes (Task 13).

\subsubsection{Methodology}
\begin{itemize}
    \item \textbf{Task 12 (Coordination):} Regex searches for patterns indicating multi-entity workflows (e.g., ``coordination with'', ``jointly with [Entity]'').
    \item \textbf{Task 13 (Legal References):} Keyword detection for statutory citations (e.g., ``pursuant to'', ``in accordance with Article'', ``statutes of'').
\end{itemize}

\subsubsection{Results}
The quantitative results, summarized in Table~\ref{tab:operational_flags}, suggest a distinct pattern in state-level diplomacy.

\begin{table}[H]
\centering
\caption{Prevalence of Operational and Legal Clauses ($N=298$)}
\label{tab:operational_flags}
\begin{tabular}{lrr}
\toprule
\textbf{Feature Detected} & \textbf{Count} & \textbf{Percentage} \\
\midrule
Task 12: Coordination with Entities & 27 & 9.1\% \\
Task 13: Legal References & 133 & 44.6\% \\
\bottomrule
\end{tabular}
\end{table}

\paragraph{Interpretation of Task 12 (Coordination):}
Most agreements returned \texttt{False} (90.9\%). This indicates that the documents are primarily \textbf{bilateral}, focusing on the direct relationship between the two signatories rather than establishing a complex network of coordination with third-party entities or agencies.

\paragraph{Interpretation of Task 13 (Legal References):}
Conversely, nearly half of the agreements (44.6\%) returned \texttt{True} for legal references. The frequent use of phrases like ``pursuant to'' proves that these are not merely ``handshake deals'' but are formally anchored in existing legal codes or prior legislative documents.



\begin{center}
\small
\setlength{\tabcolsep}{3pt}
\captionsetup{type=table}
\caption{Representative examples of clauses indicating evaluation or review of implementation automatically extracted by the system (Task~11).}
\label{tab:task11_top_examples}

\begin{tabularx}{\columnwidth}{p{0.3\columnwidth} p{0.24\columnwidth} X}
\toprule
\textbf{Document} & \textbf{Eval. Status} & \textbf{Extracted Evidence} \\
\midrule

Alaska\_7 & Present &
``The Parties shall hold an annual meeting to review the terms of this Agreement and its implementation.'' \\

California\_1July5 & Present &
``This report will evaluate the implementation of the program and suggest proposals for improvement.'' \\

Alaska\_9 & Present &
``The parties shall review the status and progress made under this Memorandum of Understanding at the end of the first year.'' \\

Arkansas\_1 & Present &
``The Parties shall regularly review the execution of the Action Plan developed under this Agreement.'' \\

Utah\_14 & Present &
``Representatives of the Parties will evaluate the activities and results related to program development and implementation.'' \\


\bottomrule
\end{tabularx}
\end{center}

\section{Work distribution}

\begin{table}[H]
\centering
\small
\setlength{\tabcolsep}{6pt}
\begin{tabular}{p{\dimexpr\columnwidth-5cm\relax} >{\centering\arraybackslash}p{1.5cm} >{\raggedleft\arraybackslash}p{2.5cm}}
\toprule
\textbf{Task} & \textbf{Workload Assessment [hours]} & \textbf{Asignee} \\
\midrule
Dataset preprocessing and OCR implementation & 5 & Mateusz Wiktorzak \\
\midrule
EDA & 3 & Paulina Kulczyk \\
\midrule
Project proposal document & 1 & Paulina Kulczyk \\
\midrule
SOTA analysis & 5 & Muhammad Fahim Asim, Paulina Kulczyk, Mateusz Wiktorzak, Mateusz Zagórski \\
\midrule
Tasks 1, 7, 9 & 30 & Mateusz Wiktorzak \\
\midrule
Tasks 4, 12, 13, 14 & 30 & Paulina Kulczyk \\
\midrule
Tasks 2, 3, 5 & 30 & Mateusz Zagórski \\
\midrule
Tasks 6, 8, 11 & 30 & Muhammad Fahim Asim \\
\midrule
Review 1 & 3 & Muhammad Fahim Asim \\
\midrule
Review 2 & 3 & Mateusz Zagórski \\
\midrule
Combining deliverables & 3 & Mateusz Zagórski \\
\bottomrule
\end{tabular}
\caption{Project responsibilities divided among all team members}
\label{tab:tasks}
\end{table}

\bibliographystyle{acl_natbib}
\bibliography{ref}

\end{document}
